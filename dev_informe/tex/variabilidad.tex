\section{Variabilidad}

 En relación a la complejidad algorítmica y la optimalidad del algortimo, la variabilidad de los valores de las monedas \textbf{no} afectan a los tiempos del algoritmo planteado, ya que lo único que se realiza son comparaciones entre las monedas, y adiciones sobre el acumulado.

 Sin embargo, lo que si afecta al tiempo, es la cantidad de monedas, ya que por cada moneda se debe realizar una iteración más (puesto que es un turno más del juego), aunque esto no cambia la complejidad del algoritmo, que siempre se mantiene lineal.
 
 En cuanto a la optimalidad del algortimo, la variablidad de los valores de las monedas no lo va a afectar, ya que como hemos demostrado en la sección \ref{ch:demostracion}, no importa qué valores tengan ni cuantás monedas haya, siempre se va a cumplir que Sophia va a ser la ganadora turno a turno. Es decir, siempre se va a dar el óptimo local para nuestro problema que nos va a guiar a nuestro óptimo global: que Sophia gane el juego.





