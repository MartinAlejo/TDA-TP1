\section{Conclusiones}

Pudimos observar y verificar lo siguiente:

\begin{itemize}

\item Utilizando un algoritmo greedy, pudimos resolver nuestro problema buscando iterativamente óptimos locales, hasta finalmente alcanzar un óptimo global: que Sophia gane el juego.
\item Tras realizar un análisis de complejidad, tanto haciendo un seguimiento del algoritmo planteado, como realizando pruebas empiricas, se concluyo que el algoritmo es de orden lineal: O(n).
\item Al realizar pruebas, tanto las otorgadas por la cátedra como las nuestras, se pudo llegar a la conclusión de que no importa la variabilidad de los valores de las monedas, siempre se llega al óptimo global.

\end {itemize}

En conclusión, el presente trabajo permitió afianzar los conocimientos adquiridos en la materia de una manera práctica, donde desarrollamos un algoritmo greedy para resolver el problema planteado, con una complejidad lineal, alcanzando de esta manera su óptimo global (Sophia siempre gana el juego).