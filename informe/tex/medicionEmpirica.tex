\section{Medición empírica}

Para comprobar empíricamente la complejidad \textbf{O(n)} del algoritmo, se decidió ejecutar el mismo con distintos tamaños de entrada y medir el tiempo de ejecución. Se generaron muestras de tamaño n, las cuales varían desde 1 hasta 100 millones.

Para cada muestra se registró el tiempo de ejecución, obteniendo el siguiente gráfico:

\begin{center}
    \includegraphics[scale = 0.6]{ {images/tiempoDeEjec.png} }
\end{center}

A simple vista se puede observar un crecimiento lineal. Para confirmar esto, vamos a ajustar los datos a una recta mediante cuadrados mínimos. Esto lo realizamos con Python y la función \textit{optimize.curve\_fit} de la librería \textit{scipy}.

Obtenemos que el gráfico se puede ajustar a la recta $y = 4.54x + 0.25$, con un error cuadrático medio de 0.06. Por lo tanto, podemos concluir que el algoritmo tiene una complejidad \textbf{O(n)}.

\begin{center}
    \includegraphics[scale = 0.6]{ {images/cuadradosMinimos.png} } 
\end{center}