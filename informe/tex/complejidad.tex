\section{Algoritmo planteado y complejidad}

El algoritmo que decidimos utilizar para resolver el problema, es el siguiente:

\begin{verbatim}
    def juego_monedas(monedas):
    turno = 0 # Los turnos pares son de Sophia, los impares de Mateo
    i = 0
    j = len(monedas) - 1

    acum_sophia = 0
    acum_mateo = 0
    movimientos = []
    while not (i > j):
        primera_moneda = monedas[i]
        ultima_moneda = monedas[j]
        if turno % 2 == 0:
            if primera_moneda > ultima_moneda:
                acum_sophia += primera_moneda
                i += 1
                movimientos.append("Primera moneda para Sophia")
            else:
                acum_sophia += ultima_moneda
                j -= 1
                movimientos.append("Última moneda para Sophia")
        else:
            if primera_moneda < ultima_moneda:
                acum_mateo += primera_moneda
                i += 1
                movimientos.append("Primera moneda para Mateo")
            else:
                acum_mateo += ultima_moneda
                j -= 1
                movimientos.append("Última moneda para Mateo")
        turno += 1

    return movimientos, acum_sophia, acum_mateo
\end{verbatim}

\begin {itemize}
\item Mientras hayan monedas para elegir:
    \begin {itemize}
    \item Vemos las monedas que se encuentran en los dos extremos de la fila, y las comparamos:
        \begin {itemize}
        \item Si el turno es de Sophia, se elige la moneda de mayor valor.
        \item Si el turno es de Mateo, se elige la moneda de menor valor.
        \end {itemize}
    \end {itemize}
\item Devolvemos la ganancia acumulada de Sophia y Mateo.
\end {itemize}

Lo que estamos haciendo, es recorrer toda la fila de monedas (con dos índices, uno para cada extremo), y en cada iteración, comparamos las monedas, y acumulamos la ganancia para Sophia o Mateo (según corresponda el turno), y se agrega el movimiento que se realizó, hasta que finalmente ya no quedan monedas. Por lo tanto, siendo $n$ las monedas de la fila, nuestro algoritmo es lineal: O(n), ya que solamente recorremos ese arreglo de monedas, y en cada iteración hacemos operaciones de tiempo constante: O(1).

En conclusión, la complejidad algorítmica es: \textbf{O(n)}.